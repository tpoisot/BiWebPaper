%% BioMed_Central_Tex_Template_v1.06
%%                                      %
%  bmc_article.tex            ver: 1.06 %
%                                       %

\NeedsTeXFormat{LaTeX2e}[1995/12/01]
\documentclass[10pt]{bmc_article}    

% Load packages
\usepackage{cite} % Make references as [1-4], not [1,2,3,4]
\usepackage{url}  % Formatting web addresses  
\usepackage{ifthen}  % Conditional 
\usepackage{multicol}   %Columns
\usepackage[utf8]{inputenc} %unicode support
\urlstyle{rm}

\usepackage{color}
\newcommand\rmq[1]{{\sffamily\color{blue} \emph{#1}}}

\def\includegraphic{}
\def\includegraphics{}

\setlength{\topmargin}{0.0cm}
\setlength{\textheight}{21.5cm}
\setlength{\oddsidemargin}{0cm} 
\setlength{\textwidth}{16.5cm}
\setlength{\columnsep}{0.6cm}

\newboolean{publ}

%%%%%%%%%%%%%%%%%%%%%%%%%%%%%%%%%%%%%%%%%%%%%%%%%%
%%                                              %%
%% You may change the following style settings  %%
%% Should you wish to format your article       %%
%% in a publication style for printing out and  %%
%% sharing with colleagues, but ensure that     %%
%% before submitting to BMC that the style is   %%
%% returned to the Review style setting.        %%
%%                                              %%
%%%%%%%%%%%%%%%%%%%%%%%%%%%%%%%%%%%%%%%%%%%%%%%%%%
 
\usepackage{pgfplots}
\usepackage{pgfplotstable}
\pgfplotsset{compat=newest}

%Review style settings
%\newenvironment{bmcformat}{\begin{raggedright}\baselineskip20pt\sloppy\setboolean{publ}{false}}{\end{raggedright}\baselineskip20pt\sloppy}

%Publication style settings
\newenvironment{bmcformat}{\fussy\setboolean{publ}{true}}{\fussy}

%New style setting
%\newenvironment{bmcformat}{\baselineskip20pt\sloppy\setboolean{publ}{false}}{\baselineskip20pt\sloppy}

\begin{document}
\begin{bmcformat}

\title{BiWeb : a common framework to facilitate the analysis of bipartite networks}
 
\author{Timoth\'ee Poisot\correspondingauthor$^{1,2}$%
         \email{Timoth\'ee Poisot\correspondingauthor - timothee.poisot@uqar.ca}
       \and 
        Cesar Flores$^3$%
         \email{Cesar Flores - \textcolor{red}{cesar.flores@gatech.edu}}%
       \and 
        Sergi Valverde$^4$%
         \email{Sergi Valverde - \textcolor{red}{svalver@gmail.com}}%
         \and
        Joshua S Weitz$^{3,5}$%
         \email{Joshua S Weitz - \textcolor{red}{jsweitz@gatech.edu}}%
      }
      

\address{%
    \iid(1) UQAR\\
    \iid(2) Qu\'ebec Centre for Biodiversity Science\\
    \iid(3) \textcolor{red}{School of Physics, Georgia Institute of Technology, Atlanta, GA 30332}\\
    \iid(4) \textcolor{red}{Universitat Pompeu Fabra, Barcelona, Spain}\\
    \iid(5) \textcolor{red}{School of Biology, Georgia Institute of Technology, Atlanta, GA 30332}    
}%

\maketitle

%%%%%%%%%%%%%%%%%%%%%%%%%%%%%%%%%%%%%%%%%%%%%%
%%                                          %%
%% The Abstract begins here                 %%
%%                                          %%  
%% Please refer to the Instructions for     %%
%% authors on http://www.biomedcentral.com  %%
%% and include the section headings         %%
%% accordingly for your article type.       %%   
%%                                          %%
%%%%%%%%%%%%%%%%%%%%%%%%%%%%%%%%%%%%%%%%%%%%%%


\begin{abstract}
  Almost all ecological interactions can be represented as networks. The
  structure of this network holds much information about the ecological
  mechanisms acting in communities, and can help predict their evolutionary
  responses. However, there is a large number of different metrics for the
  same property, which makes it difficult for people to get started with these
  analyses. In this paper, we describe two open-source packages, one in
  Octave/MatLab and one in Python, to perform the most standard analyses in a
  straigthforward way.
\end{abstract}

\ifthenelse{\boolean{publ}}{\begin{multicols}{2}}{}

%%%%%%%%%%%%%%%%
%% Background %%
%%

\section*{Background}

Viewing species interaction as a network is an efficient way to reduce their
inherent complexity to an object that is mathematically analyzable
\cite{Proulx2005,Dunne2006}. Among them, bipartite graphs \cite{Chartrand1985}
are especially suited to the analysis of systems in which two distinct sets of
organisms are interacting. Although an abundant literature on plant--pollinators
systems emerged around the use of these networks
\cite{Bascompte2007,Stouffer2011,Bascompte2003,Bastolla2009,Joppa2009}, they
are with a few exceptions \cite{Thebault2008,Thebault2010} less frequently
used to analyze antagonistic interactions \cite{Poulin2010}. Recent work
emphasized that viewing antagonistic interactions as being bipartite networks
allowed to look at emerging properties of these systems, including how they
reacted to changes in environmental quality
\cite{PoisotBL2011,PoisotPRSLB2011}, and how their structure relates to
ecological and evolutionary mechanisms \cite{Flores2011}. To some extent, the
lack of widespread use of these methods can be explained by two factors.
First, thinking in terms of networks requires to gain familiarity with a new
literature and nomenclature, which can be daunting, although recent papers
made a great effort of ``translating'' the different notions from one field to
another \cite{Poulin2010}. Second, the analysis of bipartite networks can be
an intense task from a computational point of view, and out of the multiple
methods existing to measure a given property, not all are publicly available
or easily usable. Additionaly, several groups often independently developed
different metrics for the same property of a network, and choosing the most
robust or meaningful one is a difficult task for peopke unfamiliar with the
litterature. The goal of this paper is to provide a toolbox for ecologists
wanting to analyze bipartite networks. We propose an implementation of this
toolbox in two languages, Python and MatLab.

\section*{Implementation}

We propose two alternative implementations of the toolbox, \texttt{biweb-ml}
(MatLab/Octave) and \texttt{biweb-py} (Python). Our toolbox allows the
estimation of nestedness and modularity, using two robust estimators, along
with ways of network visualization allowing to reflect both properties. It is
assumed that data come in matrix shape, with organisms of the upper level as
rows, and organisms of the lower levels as columns. The matrix can be filled
with either 0 and 1 (adjacency matrix), or with numerical values indicating
the strength of the pairwise interaction: $\mathbf{M}_{ij}$ is the strength of
the interaction between species $i$ and species $j$. However, some of the
available analysis work only with boolean matrices (i.e. nestedness,
modularity). In these cases the boolean version of $\mathbf{M}$ is used. For
any kind of matrix input, all rows and columns that contain only zeros are
deleted before any type of analysis.

For each property, as detailed in the following paragraphs, only one measure
was selected. This is intended to facilitate the analysis, as while
contrasting the results of two different metrics can be interesting in some
situations, it risks confusings people not interested by network theory. The
metrics to include were selected based on a litterature review and tests using
datasets either public or gathered by the authors.

\subsection*{Measures}

Nestedness is measured using \textsc{nodf} \cite{Almeida-Neto2008}, which is a
robust algorithm, insensitive to network shape (i.e. asymmetry in the number
of species at each level) and shape (i.e. number of species in the whole
network). Nestedness is measured for the whole network, and separately for
each of the levels.

Bipartite modularity is estimated using the \textsc{lp-brim} method
\cite{Liu2010a}, which optimizes Barber's modularity
\cite{Barber2007,Barber2008} in a short time. With this method, best results
are achieved by selecting the optimal solution of a large number of runs. In
addition to reporting Barber's $Q_{bip}$, we propose an additional measure
termed the realized modularity, noted $Q_{R}$. This measure reflects the
\emph{a posteriori} goodness of the division of the networks in modules, and
is calculated by dividing the number of interactions established by species
belonging to the same module by the total number of interactions in the
network. If all links are established by species belonging to the same module,
then $Q_{R} = 1$, reflecting the perfect modularity. If there are as many
links established between and within modules, then $Q_{R}$ takes its minimal
value of $1/2$. Note that due to this, $Q_{R}$ can be ranged in $[0,1]$, with
$Q'_{R} = 2\times Q_{R}-1$.

We also implement several ``species-level'' metrics. These include generalitty
and vulnerability \cite{Schoener1979}, specificity as measured with (\emph{PDI}) and without
(\emph{RR}) link strength \cite{PoisotMEE2011}, and the Species-
Specialization Index \cite{Julliard2006}. In addition, when the analysis to
measure modularity has been done, the module to which each species belong is
recorded.

\subsection*{Null models}

We propose four null models allowing to test the significancy of the value of
nestedness and modularity (see \cite{Bascompte2003,Ulrich2007,Flores2011} for
more details), working by generating random networks through a Bernoulli
process, where the probability of interactions are determined following
different rules. If $\mathbf{V}$ and $\mathbf{G}$ are vectors with the number
of interactions respectively received by lower level species, and established
by upper level species, $L$ is the number of interactions in the network, and
$l$ and $u$ are respectively the number of species at the lower and upper
levels, then the probability of two species having an interaction in the
random network, $P_{ij}$, is

\begin{description}
  \item[in model 1], $P_{ij} = L/(ul)$ -- the connectance of the network is respected, but not the number. of interactions in which each species is involved.  \item[in model 2], $P_{ij} = (G_{i}/l + V_{j}/u)/2$ -- the connectance, and the number of interactions in which each species is involved, are respected
  \item[in model 3r], $P_{ij} = G_{i}/l$ -- the connectance, and the number of interactions of species at the upper level, are respected
  \item[in model 3c], $P_{ij} = V_{j}/u$ -- the connectance, and the number of interactions of species at the lower level, are respected
\end{description}

We impose the additional constraint that in the random network, no species can
be left with no interaction at the end of the randomization process, so that
only networks of the same size and shape are generated through this process.
For each null model, the original value of each measure (\emph{e.g.}
\textsc{nodf}) is measured on the observed network, and assigned a value $N$.
For all of the replicates, the randomized value of the measure ($N'$) is
calculated. Once this is done for all replicates, we use the distribution of
absolute differences $N-N'$, and use a one-sample t-test to assess whether its
mean is significantly different from 0.

\subsection*{Visualization}

\section*{Results}

- FIG 1a convergence time number of runs null models\\
- FIG 1b convergence time number of runs modularity

- FIG 2  Different representations web

- TAB 1  species level

- TAB 1  network-level


\section*{Conclusions}

\section*{Availability \& requirements}

The following section lists the informations and modalities of use of the different packages.

\textbf{Project name:} biweb\\
\textbf{Project home:} http://www.github.com/tpoisot/biweb\\
\textbf{Operating system(s):} Platform independent\\
\textbf{Programming language:} MatLab, Octave, Python\\
\textbf{Other requirements:} Python 2.6 or higher, LaTeX, and the following packages: \emph{pp}, \emph{numpy}, \emph{scipy}, \emph{pyx}\\
\textbf{Licence:} GNU GPL\\
\textbf{Restrictions to use:} Papers using this package should cite the present publication

\bigskip

%%%%%%%%%%%%%%%%%%%%%%%%%%%%%%%%
\section*{Author's contributions}
Developped the software: TP, CF. Conceived the study: TP, CF, SV, JSW. Analyzed the results: TP, CF. All authors contributed to the redaction of the paper.    

%%%%%%%%%%%%%%%%%%%%%%%%%%%
\section*{Acknowledgements}
  \ifthenelse{\boolean{publ}}{\small}{}
  TP is supported by a MELS-FQRNT post-doctoral scholarship. The Canadian Research Chair on Continental Ecosystems Ecology provided computational support for the development of these tools.
 
%%%%%%%%%%%%%%%%%%%%%%%%%%%%%%%%%%%%%%%%%%%%%%%%%%%%%%%%%%%%%
%%                  The Bibliography                       %%
%%                                                         %%              
%%  Bmc_article.bst  will be used to                       %%
%%  create a .BBL file for submission, which includes      %%
%%  XML structured for BMC.                                %%
%%  After submission of the .TEX file,                     %%
%%  you will be prompted to submit your .BBL file.         %%
%%                                                         %%
%%                                                         %%
%%  Note that the displayed Bibliography will not          %% 
%%  necessarily be rendered by Latex exactly as specified  %%
%%  in the online Instructions for Authors.                %% 
%%                                                         %%
%%%%%%%%%%%%%%%%%%%%%%%%%%%%%%%%%%%%%%%%%%%%%%%%%%%%%%%%%%%%%

\newpage
{\ifthenelse{\boolean{publ}}{\footnotesize}{\small}
 \bibliographystyle{bmc_article}  % Style BST file
  \bibliography{library} }     % Bibliography file (usually '*.bib' ) 

%%%%%%%%%%%

\ifthenelse{\boolean{publ}}{\end{multicols}}{}

%%%%%%%%%%%%%%%%%%%%%%%%%%%%%%%%%%%
%%                               %%
%% Figures                       %%
%%                               %%
%% NB: this is for captions and  %%
%% Titles. All graphics must be  %%
%% submitted separately and NOT  %%
%% included in the Tex document  %%
%%                               %%
%%%%%%%%%%%%%%%%%%%%%%%%%%%%%%%%%%%

%%
%% Do not use \listoffigures as most will included as separate files

\section*{Figures}
  \subsection*{Figure 1 - Run time for the detection of community modules}

  We ran the ... Bluthgen dataset

  \begin{center}
    \begin{tikzpicture}
      \begin{semilogxaxis}[
        xlabel=Run time,
        ylabel={Modularity},
        legend pos = north east
        ]
        \addplot+[only marks, mark=*] table [x = t, y = q] {contime.dat};
        \addlegendentry{$Q_{bip}$};
        \addplot+[only marks, mark=*] table [x = t, y = qr] {contime.dat};
        \addlegendentry{$Q_R$};
      \end{semilogxaxis}
    \end{tikzpicture}
  \end{center}

  \subsection*{Figure 2 - Sample figure title}
      Figure legend text.

%%%%%%%%%%%%%%%%%%%%%%%%%%%%%%%%%%%
%%                               %%
%% Tables                        %%
%%                               %%
%%%%%%%%%%%%%%%%%%%%%%%%%%%%%%%%%%%

%% Use of \listoftables is discouraged.
%%
\section*{Tables}
  \subsection*{Table 1 - Null models and their interpretation}
    Here is an example of a \emph{small} table in \LaTeX\ using  
    \verb|\tabular{...}|. This is where the description of the table 
    should go. \par \mbox{}
    \par
    \mbox{
      \begin{tabular}{l l l}
      \hline
        \textbf{Model name} & \textbf{Works by} & \textbf{if significant}\\ \hline
        1 & creating a random network with equal connectance & C3 \\ \hline
        2 & ... & .. \\ \hline
        3r & ..  & .  \\ \hline
        3c & ..  & .  \\ \hline
      \end{tabular}
      }



%%%%%%%%%%%%%%%%%%%%%%%%%%%%%%%%%%%
%%                               %%
%% Additional Files              %%
%%                               %%
%%%%%%%%%%%%%%%%%%%%%%%%%%%%%%%%%%%

\section*{Additional Files}
  \subsection*{Additional file 1 --- Sample additional file title}
    Additional file descriptions text (including details of how to
    view the file, if it is in a non-standard format or the file extension).  This might
    refer to a multi-page table or a figure.

  \subsection*{Additional file 2 --- Sample additional file title}
    Additional file descriptions text.


\end{bmcformat}
\end{document}







